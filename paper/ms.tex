
\documentclass[11pt,letterpaper]{article}
% \documentclass[11pt]{report}
% \documentclass{report}
% \documentclass{book}
\usepackage[bookmarks]{hyperref}
\usepackage{amssymb,amsmath}
% \usepackage{fullpage}
\usepackage{tabulary}
\usepackage{tabularx}
\usepackage{float}
% \usepackage[margin=1.00in]{geometry}
\usepackage[margin=0.90in]{geometry}

\usepackage{caption}
\usepackage{booktabs}
\usepackage{pslatex}
\usepackage{apacite}
\usepackage{subcaption}
\usepackage{pgfplots}
\usepackage{wrapfig}
\usepackage[english]{babel}
\usepackage{lmodern}
\usepackage{setspace}
\doublespace
% \usepackage{url}
\usepackage{bigfoot}
\usepackage[export]{adjustbox}
\setlength\intextsep{0pt}

\usepackage{graphicx}

\title{Model and results for identity signaling project}

\author{{Paul E.~Smaldino and Matthew A.~Turner}}

\begin{document}
\maketitle

\section{Introduction}

When we interact with others it is often to mutually benefit from the
exchange. For instance, we shop at a store to buy things we want from one or
a group of people who want to do the hard work of shopkeeping. When we are
dissimilar from those with whom we interact, the benefit can be lessened.
For instance, you might be wearing a shirt with a political slogan and 
the shopkeeper has posted signs boasting their political affiliation. Perhaps
the shopkeeper decides to charge you more, or perhaps you feel a sense of 
resentment for giving money to someone with whom you disagree politically;
maybe the shopkeeper's signage distracted you from making all the purchases 
you needed to make. Shopping is rather inconsequential, but illustrates the issue. 
A much more critical situation is the interaction of political partisans, who
must mutually benefit one another for democracy to work.

In response to the fact that
interpersonal disliking often lowers payoffs from interaction, humans have
developed \emph{covert signaling} strategies so that, e.g.,
customer and shopkeeper can signal their political affiliations so that like-minded
people will know they are similar, but dissimilar others will be unaware
of political or identity signaling at all. Different social settings present
different signaling pressures. Sometimes it is possible to choose who we assort
with for interaction, which means the practice of homophily---selectively
interacting with similar others---is widespread. In that case, there is little
reason to signal covertly because we can preferentially 
interact with similar others, avoiding penalties for interaction with 
dissimilar others. Sometimes the penalties for
disliking one another are very high, which pressures individuals to choose
covert signaling more often in order to not be disliked. While there are
interaction costs for dislike there are benefits for the case where two, which are
also included in the model.

Here we present a model of identity signaling, social interaction,
and social learning to identify mechanisms for the evolution of the covert 
signaling strategy, as opposed to an overt signaling strategy. The precise 
definitions of overt and covert signaling are given in the Model section.
Our results are thus predictions of when covert signaling will or
will not evolve. This depends on a number of model parameters and the 
co-evolution of the \emph{receiving} strategy, which may be either \emph{generous}
or \emph{churlish}. Generous individuals do not dislike
others with whom they have not previously interacted. Churlish individuals 
dislike others with whom they have not previously interacted.

The model predicts that covert signaling evolves when homophily is low (agents
do not efficiently assort) and the costs of disliking are high. The model also 
predicts that covert signaling co-evolves with generous receiving. This is due
to the fact that covert signalers are simply ignored by those with whom
they are different. If most of the individuals are churlish receivers, there is
no benefit to being unknown since most individuals dislike unknown others. 
We also use our model to predict differences in covert signaling prevalence
among minority and majority populations, where the minority and majority all
share the same in-group traits along a number of trait dimensions, but exactly
the opposite traits. If homophily is weak and disliking costs are high, the
minority will tend to become covert signalers. However, if homophily is
sufficiently high it is better for the minority to overtly signal so they can
efficiently identify similar interaction partners. We close by finding the
conditions that promote either signaling or receiving strategies to ``invade'',
i.e.\ become prevalent after starting as just a small fraction of the population.
We find here that sometimes no signaling is better than overt signaling: 
covert signaling can evolve even when the efficiency of covert signaling is
zero, i.e.\ covert signals are not understood by any model individuals (agents).

\section{Model}

The model includes parameters to represent homophily, the penalty for 
individuals who dislike one another, the efficiency of covert signaling, and the
number of traits assumed to be important in social interaction for a model system.
These are summarized in Table~\ref{tab:params} below.

\vspace{1em}
\begin{table}[H]
  \centering
  \begin{tabular}{cl}
    Global parameter & Description \\
    \toprule 
    $K$      & Number of relevant traits determining similarity, liking, disliking, etc. \\
    $w$      & Homophily \\
    $d$      & Interaction penalty when one agent dislikes the other \\
    $\delta$ & Additional interaction penalty when both agents dislike each other \\
    $s$      & Added interaction benefit when agents are similar, as defined by their traits \\
    $S$      & Number of traits two agents must have in common to be considered similar \\
    $M$      & Number of traits used (if any) to define majority/minority populations \\
  \end{tabular}
  \caption{Global model parameters.}
  \label{tab:params}
\end{table}

\subsection{Agents}

Model individuals are \emph{agents} who 
\begin{enumerate}
  \item have static ``in-born'' traits that determine pairwise similarity or dissimilarity
  \item hold attitudes about all other agents in the system
  \item have adopted a signaling and receiving strategy
\end{enumerate}Signaling strategies
may be either \emph{covert} or \emph{overt}, and receiving strategies may be
either \emph{generous} or \emph{churlish}. Unless otherwise noted, agents are
randomly initialized to be covert or overt with equal probability. The 
proportion of agents with specific signaling or receiving strategies will be
set to different values in the invasion experiments, explained in more detail below.

Agents have $K$ traits, where $K$ is a parameter representing the cultural 
complexity of the system. In the real world, different systems may have 
different cultural complexities. For instance, in United States politics, there
is increasingly little subtlety and increasing partisanship, so $K \approx1 $.
However, if we were trying to understand scholarly positions on some research
question, we may have $K$ be very large, or $K$ could be large if we consider
agents to be, e.g., nations instead of people (as in Axelrod 1997).  

\begin{table}[H]
  \centering
  \begin{tabular}{cp{4.5in}}
    Agent property & Description \\
    \toprule
    $a_{ij}$   & Attitude of agent $i$ towards agent $j$; $a_i$ is the $N$-dimensional 
      attitude vector of agent $i$. \\
    $\tau_{ik}$ & Trait $k$ of agent $i$; $\tau_i$ is the $K$-dimensional trait vector of agent $i$. \\
    Signaling strategy & Either \emph{overt} or \emph{covert}. Overt signaling always
                         results in the communication of the signaler's traits,
                         while covert signalers only reveal their traits to 
                         ``similar'' others; dissimilar others do not receive a covert signal. \\
    Receiving strategy & Either \emph{churlish} or \emph{generous}. Churlish
                         receivers default to dislike other agents from whom they have
                         not received a signal. Generous receivers default to
                         a neutral attitude towards others from whom they have
                         not received a signal.
  \end{tabular}
  \caption{Agent attributes.}
  \label{tab:agentAttributes}
\end{table}


\subsubsection{Determining inter-agent similarity}

Each trait $j$ of agent $i$ takes a value $\tau_{ik} \in \{-1, 1\}$. $-1$ and 1
are used only to emphasize that there are two opposing traits.
Attitudes are $a_{ij} \in \{-1, 0, 1\}$ with meanings of 
disliking, neutral, or liking. Attitudes are set at the signaling and receiving
rounds, explained in more detail below. Briefly, if agent $i$ recieves a
trait signal from agent $j$, $a_{ij}$ will be 1 if the two agents have
similar traits, and $-1$ if $i$ and $j$ are sufficiently different. Similarity
is $s_{ij} = 1 - d(\tau_i, \tau_j)$ where $d(\tau_i, \tau_j)$ is the
Hamming distance between trait vectors. The Hamming distance is the fraction
of vector elements that are different. So, if $\tau_i = (1,~-1,~1)$ and
$\tau_j = (1,~-1,~-1)$, then $d(\tau_i, \tau_j) = 1/3$ and $s_{ij}=2/3$.
Two agents are ``similar'' if their similarity $s_{ij} \geq S$, where
$S$ is the similarity threshold parameter. Otherwise they are considered
dissimilar. 

If two agents are similar, this enables agents to like one another. On the
other hand, dissimilar agents may come to dislike one another if they signal
their traits to one another. As explained further in the following section on
model dynamics, inter-agent attitudes determine the probability agents will
interact; the payoff received when a particular dyad interacts; and the
probability that one agent chooses another as a teacher.

\subsection{Dynamics}

Each model iteration consist of three stages. First, agents signal their identities
to some fraction of all other agents. It is at this step that agent attitudes
$a_{ij}$ are set. Whether or not an agent receives a signal depends on the
efficiency of overt and covert signaling, $E_{ov}$ and $E_{cov}$, respectively.
\emph{Efficiency} is the proportion of agents in the population that receive
a given agent's signal depending on whether the agent is an overt or covert
signaler.  Signals are sent and received once every model iteration. 
If agent $i$ does receive a signal that agent updates its attitudes about
signaler $j$, then $i$ updates its attitudes about $j$ ($a_{ij}$), according to
the process outlined above.

Then agents then
interact at most $N_I$ times with other agents. In each of the possible 
$N_I$ interactions, agents first match with one agent selected randomly from
the population. The probability these agents interact after matching depends
on the homophily, $w$, and the attitudes each agent has for the other. 
If a pair interacts, their payoff is determined by whether the agents are
similar, whether one dislikes the other, and the penalty for disliking,
$d$\footnote{(WILL INTRODUCE $\delta$ WITH A REFERENCE TO THE SI IN SUBSUBSEC
BELOW)}. The distinction between similarity and disliking is subtle, 
but necessary to include, as we explain in more detail below. After all
$N_I$ interaction rounds, each agent chooses a teacher to potentially learn
from. ``Learning'' is simply the adoption of either the signaling or
receiving strategy of the teacher by the learner---which strategy type is 
learned is chosen at random. 

After several hundred iterations the prevalence of each signaling and
receiving strategy reaches an equilibrium (see SI dynamics plots). 


\subsubsection{Signaling and receiving}

At the signaling and receiving step, agents form attitudes about one another if
they receive a signal from 

\subsubsection{Interaction}

The two agents interact with different probabilities
depending on their attitudes about one another, and depending on the 
assumed homophily level. If  Note that as $w \rightarrow 0.5$,
agents never interact with agents they dislike and always interact with 
agents they like.

\begin{equation}
  \Pr(\text{$i$ interacts with $j$}) = \frac{1 + w(a_{ij} + a_{ji})}{2}
\end{equation}
\noindent

\subsubsection{Payoff}

Payoffs are determined based on the similarity bonus, $s$, and the disliking
penalty $d$\footnote{In the SI we consider the case where the disliking penalty 
when both agents dislike each other is different from $d$. We call this $\delta$
and let it vary from 0 to $2d$}. 

\begin{table}[h]
  \centering
  \begin{tabular}{cccc}
    Similar? & $a_{ij}$ & $a_{ji}$ & Payoff $\pi_i = \pi_j$ \\
    \toprule
    Yes   & 1 & 1   & $1 + s$ \\
    Yes   & 1 & 0   & $1 + s$ \\
    Yes   & 1 & -1  & $1 + s - d$ \\
    Yes   & 0 & 0   & $1 + s$ \\
    Yes   & -1 & 0  & $1 + s - d$ \\
    Yes   & -1 & -1 & $1 + s - 2d$ \\
    \midrule
    No    & 1 & -1  & $1 + s - d$ \\
    No    & 0 & 0   & $1$ \\
    No    & -1 & 0  & $1 - d$ \\
    No    & -1 & -1 & $1 - 2d$ \\
  \end{tabular}
  \caption{Interaction payoff matrix. Certain conditions are impossible, and so are not
  shown. For instance, it is necessary for two agents to be similar for 
  one agent to like another agent (i.e.\ have $a_{ij} = 1$). Note that
  due to churlish receiving, two agents may be similar, but one dislikes the
  other while the other likes the one. There could be agents who dislike one
  another due to churlish receiving if they have never signaled each other.}
\end{table}

\subsubsection{Social learning}

All agents select another agent in the population to be a teacher.
Teacher selection is random, where the probability an agent selects another
agent is proportional to the probability of interaction given in 
Equation~\ref{eq:interactionProb}, normalized by the sum of all interaction
probabilities,

\begin{equation}
  \Pr(\text{$i$ chooses teacher $j$}) = 
    \frac{1 + w(a_{ij} + a_{ji})}{\sum_{j \neq i}(1 + w(a_{ij} + a_{ji}))}
\end{equation}
% \noindent

The 
learner agent probabilistically adopts its teacher's signaling or receiving 
strategy---which type of strategy might be learned is determined by a coin flip. 
The probability that the learner adopts the teacher's strategy is set 
calculated by the logistic sigmoid function applied to the quotient of 
teacher and learner payoffs accumulated in one
signaling/receiving-interaction/payoff round.
Let $\pi_{i}$ be the accumulated payoff of agent $i$, and assume $i$ learns
from $j$. Set $p_{ij} = \pi_{j} / \pi_i$. 
The probability the learner will adopt the teacher's strategy is

\begin{equation}
  \Pr(\text{$i$ learns from $j$}) = \frac{1}{1 + e^{-\beta(p_{ij} - \alpha)}}.
\end{equation}
\noindent
$\alpha$ and $\beta$ are parameters shift and scale the sigmoid curve, respectively. 
$\alpha$ is the size of $p_{ij}$ such that $\Pr(\text{$i$ learns from $j$}) = 0.5$. % $P(\text{$i$ learns from $j$}) = 0.5$.
For example, if we set $\alpha = 2$, then the teacher's accumulated payoff
must be twice that of the learner's for a 50\% chance that the learner 
adopts the teacher's strategy. 
For the main results we set $\alpha=1.25$. $\beta$ sets how quickly 
$\Pr(\text{$i$ learns from $j$})$ increases overall. Larger $\beta$ means that
if $p_{ij} < 0.5$ then $\Pr(\text{$i$ learns from $j$})$ is closer to 0 than smaller $\beta$. 
$\Pr(\text{$i$ learns from $j$})$ is closer to 1 for larger than for smaller $\beta$. For the main
results we set $\beta = 12$. These settings are rather arbitrary, (SO WE NEED
TO PROVIDE SENSITIVITY CHECKS IN THE SUPPLEMENT).


\subsection{Computational experiment design}

\subsection{Outcome measures}

We are primarily interested in the prevalence of covert signaling at time
$t$. Usually, $t=T$, where $T$ is the number of iterations chosen for our 
computational experiments. The prevalence of overt signalers is the complement 
of the prevalence of covert signalers. One important result shows that the
prevalence of covert signalers at $t=T$ is positively correlated with the
prevalence of generous receivers at $t=T$.  

\begin{table}[H]
  \centering
  \begin{tabular}{rp{3in}} %p{5in}}
    Symbol & Measure \\ %& Description \\
    \toprule
      $\rho_{cov}$ & Prevalence of covert signalers\\
    $1 - \rho_{cov} = \rho_{ov}$ & Prevalence of overt signalers  \\
     $\rho_{gen}$ & Prevalence of generous receivers \\
    $1 - \rho_{gen} = \rho_{ch}$ & Prevalence of churlish receivers \\
    $\rho_i^{minor}$, $\rho_i^{maj}$ & Prevalence of strategy $i \in \{cov, ov, gen, ch\}$ amongst minority/majority \\
    $\Pr(\text{$i$ invades})$ & Probability strategy $i$ invades 
  \end{tabular}
  \caption{Summary of outcome measures. To indicate any of these values at
  a particular time, we add an additional subscript. For example, $\rho_{cov,t}^{maj}$
  is the prevalence of covert signalers among the majority at time $t$.}
  \label{tab:outcomeMeasures}
\end{table}





\section{Results}

\subsection{Evolution of covert signaling...}

\subsubsection{...under various homophily and disliking scenarios}

\subsubsection{...as a function of covert signaling efficiency}

\subsubsection{Co-evolution of covert signaling and generous receiving}

Theoretically, conditions favorable for covert
signalers should also be favorable for generous receivers (TRUE??). Furthermore, 
an increase in covert signalers increases the pressure towards generous receiving.
With many covert signalers it is likely an agent will get no signal on a 
turn because they were receiving a signal from a covert signaler. Disliking
incurs a penalty, so it is better to remain neutral.

Our computational results support this theoretical claim. As the heatmaps
below reveal... In the SI we show regressions of $\rho_{cov,T} \sim \rho_{gen,T}$.

\subsection{Minority-majority dynamics}

\subsection{Covert signaling invasion}

\subsection{Non-signaling as an adaptive strategy}

% \bibliographystyle{apacite}

% \setlength{\bibleftmargin}{.125in}
% \setlength{\bibindent}{-\bibleftmargin}

% \bibliography{/Users/mt/workspace/papers/library.bib}

\appendix

\section{Additional analyses}

\subsection{Temporal dynamics}

\subsection{Correlations between churlish receiving and covert signaling}

\subsection{$d \neq \delta$}

Vary $\delta$ from 0 to $2d$.

\subsection{Minority dynamics} 

\subsubsection{More traits ($K=9$ and $M=4$) and different similarity threshold}

\subsubsection{Varying minority group size}

\subsection{Invasion of other strategies}


\section{Model implementation}

In order to aid evaluation of our model code, we provide notes below on 
our model implementation. The code is freely available on GitHub at
\url{https://github.com/mt-digital/identity-signaling}.

\end{document}
