
\documentclass[11pt,letterpaper]{article}
% \documentclass[11pt]{report}
% \documentclass{report}
% \documentclass{book}
\usepackage[bookmarks]{hyperref}
\usepackage{amssymb,amsmath}
% \usepackage{fullpage}
\usepackage{tabulary}
\usepackage{tabularx}
\usepackage{float}
% \usepackage[margin=1.00in]{geometry}
\usepackage[margin=0.90in]{geometry}

\usepackage{caption}
\usepackage{booktabs}
\usepackage{pslatex}
\usepackage{apacite}
\usepackage{subcaption}
\usepackage{pgfplots}
\usepackage{wrapfig}
\usepackage[english]{babel}
\usepackage{lmodern}
\usepackage{setspace}
\doublespace
% \usepackage{url}
\usepackage{bigfoot}
\usepackage[export]{adjustbox}
\setlength\intextsep{0pt}

\usepackage{graphicx}

\title{Model and results for identity signaling project}

\author{{Paul E.~Smaldino and Matthew A.~Turner}}

\begin{document}
\maketitle

\section{Introduction}

When we interact with others it is often to mutually benefit from the
exchange. For instance, we shop at a store to buy things we want from one or
a group of people who want to do the hard work of shopkeeping. When we are
dissimilar from those with whom we interact, the benefit can be lessened.
For instance, you might be wearing a shirt with a political slogan and 
the shopkeeper has posted signs boasting their political affiliation. Perhaps
the shopkeeper decides to charge you more, or perhaps you feel a sense of 
resentment for giving money to someone with whom you disagree politically;
maybe the shopkeeper's signage distracted you from making all the purchases 
you needed to make. Shopping is rather inconsequential, but illustrates the issue. 
A much more critical situation is the interaction of political partisans, who
must mutually benefit one another for democracy to work.

In response to the fact that
interpersonal disliking often lowers payoffs from interaction, humans have
developed \emph{covert signaling} strategies so that, e.g.,
customer and shopkeeper can signal their political affiliations so that like-minded
people will know they are similar, but dissimilar others will be unaware
of political or identity signaling at all. Different social settings present
different signaling pressures. Sometimes it is possible to choose who we assort
with for interaction, which means the practice of homophily---selectively
interacting with similar others---is widespread. In that case, there is little
reason to signal covertly because we can preferentially 
interact with similar others, avoiding penalties for interaction with 
dissimilar others. Sometimes the penalties for
disliking one another are very high, which pressures individuals to choose
covert signaling more often in order to not be disliked. While there are
interaction costs for dislike there are benefits for the case where two, which are
also included in the model.

Here we present a model of identity signaling, social interaction,
and social learning to identify mechanisms for the evolution of the covert 
signaling strategy, as opposed to an overt signaling strategy. The precise 
definitions of overt and covert signaling are given in the Model section.
Our results are thus predictions of when covert signaling will or
will not evolve. This depends on a number of model parameters and the 
co-evolution of the \emph{receiving} strategy, which may be either \emph{generous}
or \emph{churlish}. Generous individuals do not dislike
others with whom they have not previously interacted. Churlish individuals 
dislike others with whom they have not previously interacted.

The model predicts that covert signaling evolves when homophily is low (agents
do not efficiently assort) and the costs of disliking are high. The model also 
predicts that covert signaling co-evolves with generous receiving. This is due
to the fact that covert signalers are simply ignored by those with whom
they are different. If most of the individuals are churlish receivers, there is
no benefit to being unknown since most individuals dislike unknown others. 
We also use our model to predict differences in covert signaling prevalence
among minority and majority populations, where the minority and majority all
share the same in-group traits along a number of trait dimensions, but exactly
the opposite traits. If homophily is weak and disliking costs are high, the
minority will tend to become covert signalers. However, if homophily is
sufficiently high it is better for the minority to overtly signal so they can
efficiently identify similar interaction partners. We close by finding the
conditions that promote either signaling or receiving strategies to ``invade'',
i.e.\ become prevalent after starting as just a small fraction of the population.
We find here that sometimes no signaling is better than overt signaling: 
covert signaling can evolve even when the efficiency of covert signaling is
zero, i.e.\ covert signals are not understood by any model individuals (agents).

\section{Model}

The model includes parameters to represent homophily, the penalty for 
individuals who dislike one another, the efficiency of covert signaling, and the
number of traits assumed to be important in social interaction for a model system.
These are summarized in Table~\ref{tab:params} below.

\vspace{1em}
\begin{table}[H]
  \centering
  \begin{tabular}{cl}
    Variable/parameter & Definition \\
    \toprule 
    $a_{ij}$   & Attitude of agent $i$ towards agent $j$; $a_i$ is the $N$-dimensional 
      attitude vector of agent $i$ \\
    $\tau_{ij}$ & Trait $j$ of agent $i$; $\tau_i$ is the $K$-dimensional trait vector of agent $i$ \\
    $K$      & Number of relevant traits determining similarity, liking, disliking, etc. \\
    $w$      & Homophily \\
    $d$      & Interaction penalty when one agent dislikes the other \\
    $\delta$ & Additional interaction penalty when both agents dislike each other \\
    $s$      & Added interaction benefit when agents are similar, as defined by their traits \\
    $S$      & Number of traits two agents must have in common to be considered similar \\
    $M$      & Number of traits used (if any) to define majority/minority populations \\
  \end{tabular}
  \caption{Model parameters.}
  \label{tab:params}
\end{table}

\subsection{Agents}

Model individuals are agents who 
\begin{enumerate}
  \item have static ``in-born'' traits that determine pairwise similarity or dissimilarity
  \item hold attitudes about all other agents in the system
  \item have adopted a signaling and receiving strategy
\end{enumerate}Signaling strategies
may be either \emph{covert} or \emph{overt}, and receiving strategies may be
either \emph{generous} or \emph{churlish}. Unless otherwise noted, agents are
randomly initialized to be covert or overt with equal probability. The 
proportion of agents with specific signaling or receiving strategies will be
set to different values in the invasion experiments, explained in more detail below.

Each trait is represented arbitrarily by $\tau_{ij} \in \{-1, 1\}$ 
(MAYBE USE 'A' and 'B' INSTEAD SINCE THEY REALLY ARE CATEGORICAL?). 
Attitudes are $a_{ij} \in \{-1, 0, 1\}$ with meanings of 
disliking, neutral, or liking. 

\subsection{Signaling and receiving}

\subsection{Interaction and payoff}

\subsection{Social learning}

\subsection{Computational experiment design}



Homophily not only determines 


% The model is 

% \begin{figure}[htpb]
% \begin{center}
% \caption{Model flow chart with relevant equations.}
% \label{fig:yo}
% \begin{tikzpicture}[]
%   \node[draw] at (0, 0) {}; 
% \end{tikzpicture}
% \end{center}
% \end{figure}


\section{Results}

\subsection{Evolution of covert signaling...}

\subsubsection{...under various homophily and disliking scenarios}

\subsubsection{...as a function of covert signaling efficiency}

\subsubsection{Co-evolution of covert signaling and generous receiving}

\subsection{Minority-majority dynamics}

\subsection{Covert signaling invasion}

\subsection{Non-signaling as an adaptive strategy}

% \bibliographystyle{apacite}

% \setlength{\bibleftmargin}{.125in}
% \setlength{\bibindent}{-\bibleftmargin}

% \bibliography{/Users/mt/workspace/papers/library.bib}

\appendix

\section{Additional analyses}

\subsection{Temporal dynamics}

\subsection{Correlations between churlish receiving and covert signaling}

\subsection{$d \neq \delta$}

Vary $\delta$ from 0 to $2d$.

\subsection{Minority dynamics} 

\subsubsection{More traits ($K=9$ and $M=4$) and different similarity threshold}

\subsubsection{Varying minority group size}

\subsection{Invasion of other strategies}



\end{document}
